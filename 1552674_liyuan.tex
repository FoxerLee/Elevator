\documentclass{article}
\usepackage{CJK}
\usepackage{indentfirst}
\usepackage{anysize}
\usepackage{graphicx}
\usepackage{subfigure}
\usepackage{array}
\usepackage{makecell}
\usepackage{url}

%标题缩进
\usepackage[bf, small]{titlesec}
    \titleformat{\section}{\bf\large}{\thesection.\,}{0.24em}{}
    \titlespacing{\section}{0cm}{*1.5}{*1.1}
    \titleformat{\subsection}{\bf}{\thesubsection.\enspace}{0.5em}{}
    \titlespacing{\subsection}{15pt}{*1.5}{*1.1}
    \titleformat{\subsubsection}{}{\thesubsubsection.\,}{0.24em}{}
    \titlespacing{\subsubsection}{30pt}{*1.5}{*1.1}

\linespread{1.5}


\marginsize{3.5cm}{3.5cm}{2cm}{2cm}
\setlength{\parindent}{2em}

\newcolumntype{L}[1]{>{\vspace{0.5em}\begin{minipage}{#1}\raggedright\let\newline\\
\arraybackslash\hspace{0pt}}m{#1}<{\end{minipage}\vspace{0.5em}}}
\newcolumntype{R}[1]{>{\vspace{0.5em}\begin{minipage}{#1}\raggedleft\let\newline\\
\arraybackslash\hspace{0pt}}m{#1}<{\end{minipage}\vspace{0.5em}}}
\newcolumntype{C}[1]{>{\vspace{0.5em}\begin{minipage}{#1}\centering\let\newline\\
\arraybackslash\hspace{0pt}}m{#1}<{\end{minipage}\vspace{0.5em}}}


%使得图片显示对应章节
\renewcommand\thefigure{\thesection.\arabic{figure}}
\makeatletter
\@addtoreset{figure}{section}
\makeatother

\begin{document} 
\begin{CJK}{UTF8}{gbsn}

\newcommand*{\titleGP}{\begingroup % Create the command for including the title page in the document
\centering % Center all text
\vspace*{\baselineskip} % White space at the top of the page

\rule{\textwidth}{1.6pt}\vspace*{-\baselineskip}\vspace*{2pt} % Thick horizontal line
\rule{\textwidth}{0.4pt}\\[\baselineskip] % Thin horizontal line

{\LARGE 基于多线程的电梯调度系统 \\ \vspace{2em} \begin{large} 操作系统课程设计 \end{large}}\\[0.2\baselineskip] % Title

\rule{\textwidth}{0.4pt}\vspace*{-\baselineskip}\vspace{3.2pt} % Thin horizontal line
\rule{\textwidth}{1.6pt}\\[\baselineskip] % Thick horizontal line

\scshape % Small caps
%利用操作系统中的多线程思想,自我实现电梯调度算法 \\[\baselineskip] % Tagline(s) or further description
operating system,  Spring 2017\par % Location and year

\vspace*{2\baselineskip} % Whitespace between location/year and editors

 By \\[\baselineskip]
{\Large1552674 李源 \par} % Editor list


\vfill % Whitespace between editor names and publisher logo

{\itshape Tongji University \\ School of Software Engineering \par}

\endgroup}


\titleGP % This command includes the title page
\clearpage
\tableofcontents
\clearpage

\section{项目背景}
\subsection{项目需求}
某一栋楼共有20层,五部互相关联的电梯,请基于线程的思想,模拟实现一个电梯调度的程序。程序中的功能包含以下部分:

(1)电梯内有楼层选择按钮、警报按钮;

(2)电梯外每一层有上行按钮、下行按钮;

(3)对于每一部电梯,有控制其工作、不工作的两个按钮;

(4)每一部电梯,能够显示所在楼层、运行状态、开关门状态。


\subsection{项目目的}

(1)通过控制电梯调度,实现操作系统调度过程;

(2)学习特定环境下多线程编程方法;

(3)学习调度算法。

\vspace{3em}

\section{项目分析}
根据实际情况即项目需求,可将电梯的调度分为两个部分,本文档首先对两个部分的需求分别进行分析,构建算法,随后再考虑两个部分如何实现联系。两个部分分别为:

(1)人在电梯内部,按下楼层,电梯将人送到指定楼层;

(2)人在电梯外某一层,按下向上或向下按钮,选择一部电梯来接人。

\subsection{电梯内部}
当人处在电梯内部时,可以



\end{CJK}
\end{document}